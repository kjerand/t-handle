\section{Innledning}
Dette er rapporten til det numeriske prosjektet i TDAT3024 hvor vi skal analysere rotasjonen til en t-nøkkel. T-nøkkelen er modellert som to sylindere festet til hverandre, vi ser på det som et stivt legeme. Vi skal numerisk tilnærme hvordan den vil rotere med ulik vinkelfart og fastslå om resultatene er rimelige.
\newline\newline
De konkrete oppgavene som skal løses innebærer blant annet å beregne treghetsmomentet til nøkkelen, og den kinetiske energien. Vi skal også implementere diverse numeriske metoder som brukes til å løse differiensiallikninger. Metodene vi vil bruke er Runge Kutta-metoder, slik som Eulers metode, RK4 og RKF45. Disse vil vi trenge for å beregne rotasjonen til legemet. Vi skal også beregne rotasjonen eksakt når vi forenkler legemet til å være en kule som kun har vinkelfart om én akse. Der det er naturlig skal vi også lage animasjoner av resultatene så vi kan se at rotasjonen virker fornuftig, og beregne energien i systemet over tid for å forsikre oss om at resultatene er til å stole på.
\newline\newline
Vi skal implementere alle metodene i Python 3 ved hjelp av biblioteket NumPy. Videre skal vi bruke matplotlib og PyOpenGL til å vise frem grafer som relaterer til nøkkelen, og animere selve nøkkelen over tid.
\newline\newline
Vi kommer til å bruke forskjellige numeriske metoder for å undersøke bevegelsene til det stive legemet. Spesifikt kommer vi til å se på Eulers-metode, RK4-metoden og RKF45-metoden. Ved å benytte oss av forskjellige metoder ønsker vi å komme frem til hvilken metode som gir oss den beste modellen for denne problemstillingen. Vi skal også gi et matematisk bevis for hvorfor Mellomakse-teoremet (Intermediate Axis Theorem) er sann, og hvordan det teoremet påvirker rotasjonen til t-nøkkelen når den spinner om en viss akse.
\newline\newline
Vi løser denne problemstillingen for å få en bedre forståelse av matematikken bak det fysiske fenomenet og samtidig finne ut hvordan vi best kan modellere det. For å oppnå dette ønsker vi å finne ut hvilke numeriske metoder som er best egnet til å modellere i forskjellige situasjoner og hvorfor. Dette inkluderer nøyaktighet av modell samt kompleksiteten til metoden og kjøretid. 