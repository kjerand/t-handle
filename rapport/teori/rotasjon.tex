\graphicspath{ {./bilder/} }

\subsection{Rotasjon av stive legemer}

\subsubsection{Rotasjonell kinematikk}
Når et stivt legeme roteres rundt en stasjonær akse (vanligvis $z$-aksen), så er posisjonen til det stive legemet beskrevet av vinkelen $\theta.$\newline\newline
\begin{center}
\includegraphics{rapport/teori/bilder/theta.png}\newline\newline
\end{center}
Her ser vi et eksempel på et stivt legeme som roterer rundt en stasjonær akse \cite{FYSIKK:1}. Som vi kan se er legemets posisjon beskrevet av $\theta.$\newline\newline
Vinkelfarten $\omega_z$ er den tidsderiverte av $\theta$, og vinkelakselerasjonen $\alpha_z$ er den tidsderiverte av $\omega_z$, eller den andrederiverte av $\theta.$
\begin{equation}
\label{eqn:first}
    \omega_z=\lim_{\Delta t\rightarrow0}{\frac{\Delta\theta}{\Delta t}}=\frac{d\theta}{dt}
\end{equation}
\begin{equation}
    \alpha_z=\lim_{\Delta t\rightarrow0}{\frac{\Delta\omega_z}{\Delta t}}=\frac{d\omega_z}{dt}
\end{equation}
Hvis vinkelakselerasjonen er konstant, så er $\theta$, $\omega_z$, og $\alpha_z$ relatert med kinematikk-ligningene som er ekvivalente med de som gjelder for lineær kinematikk:
\begin{equation}
    \theta=\theta_0+\omega_{0z}t+\frac{1}{2}\alpha_zt^2
\end{equation}
\begin{equation}
    \theta-\theta_0=\frac{1}{2}\left(\omega_{0z}+\omega_z\right)^2
\end{equation}
\begin{equation}
    \omega_z=\omega_{0z}+\alpha_zt
\end{equation}
\begin{equation}
    \omega_z^2=\omega_{0z}^2+2\alpha_z(\theta-\theta_0)
\end{equation}

\subsubsection{Forholdet mellom lineær og rotasjonell kinematikk}
For både lineær kinematikk og rotasjonell kinematikk gjelder følgende:\newline\newline
\begin{itemize}
    \item Vinkelhastigheten $\omega$ til et stivt legeme er lengden til vinkelfarten.
    \item Endringsraten til $\omega$ er $\alpha=\frac{d\omega}{dt}.$
    \item For en partikkel i legemet en distanse $r$ fra rotasjonsaksen, er farten og komponentene til $\vec{a}$ relatert til $\omega$ og $\alpha$ på følgende måte:
\end{itemize}
\begin{equation}
    v=r\omega
\end{equation}
\begin{equation}
    a_{\text{tan}}=\frac{dv}{dt}=r\frac{d\omega}{dt}=r\alpha
\end{equation}
\begin{equation}
    a_{\text{rad}}=\frac{v^2}{r}=\omega^2r
\end{equation}
\begin{center}
\includegraphics{rapport/teori/bilder/tang.png}\newline
\end{center}
Her er $a_\text{rad}$ og $a_\text{rad}$ henholdsvis den radielle komponenten og den tangentielle komponenten til akseleasjonsvektoren \cite{FYSIKK:1}

\subsubsection{Treghetsmoment og rotasjonell kinetisk energi}
Treghetsmomentet $I$ til et legeme om en gitt akse er et mål på legemets rotasjonelle treget: Jo større verdien til $I$ er, jo vanskeligere er det å endre tilstanden til rotasjonen. Treghetsmomentet kan bli skrevet som en sum over massen til partiklene, $m_i$, som utgjør hele legemet. Hver partikkel har en distanse $r_i$ fra aksen.
\begin{equation}
    I=m_1r_1^2+m_2r_2^2+\dots=\sum_i{m_i_r_i^2}
\end{equation}
Den rotasjonelle kinetiske energien til et stivt legeme som roterer om en akse avhenger av vinkelhastigheten $\omega$ og treghetsmomentet $I$ for rotasjonsaksen
\begin{equation}
    K=\frac{1}{2}I\omega^2
\end{equation}
\subsubsection{Beregning av treghetsmoment}
Parallell akse teoremet beskriver forholdet mellom treghetsmomentet til et stivt legeme med masse $M$ om to parallelle akser. Den ene aksen er aksen gjennom massesenteret, og den andre aksen er aksen en distanse $d$ fra den første aksen
\begin{center}
\includegraphics{rapport/teori/bilder/parallell.png}
\end{center}
Her ser vi et stivt legeme med masse $M$, hvor den ene aksen går gjennom massesenteret. Her er treghetsmomentet $I_\text{cm}$. Den andre aksen ligger en distanse $d$ fra første akse, og her er treghetsmomentet $I_p.$\cite{FYSIKK:1} Parallell akse teoremet gir oss at
\begin{equation}
    I_p=I_\text{cm}+Md^2
\end{equation}
Hvis det stive legemet har en kontinuerlig distribusjon av masse, så kan treghetsmomentet regnes ut med integrasjon. Dette kommer fra likning $(10)$, der treghetsmomentet er en sum over massen til alle partiklene i det stive legemet. Da har vi at
\begin{equation}
    I=\int{r^2\,dm}
\end{equation}
Hvis det stive legemet er endimensjonal, så kan vi sette $x$-aksen langs lengden, og beskrive $dm$ som et inkrement $dx.$ For en tredimensjonal legeme er det enklere å beskrive $dm$ med volumnet $dV$ og massetettheten $\rho$ til det stive legemet. Vi har at
\begin{equation}
    \rho=\frac{dm}{dV}
\end{equation}
Dette gir oss
\begin{equation}
    I=\int{r^2\rho\,dV}
\end{equation}
Likningen ovenfor forteller oss at treghetsmomentet avhenger av hvordan massetettheten varierer i volumnet til det stive legemet. Hvis legemet har uniform massetetthet, så kan vi skrive om likning $(15)$
\begin{equation}
    I=\rho\int{r^2\,dV}
\end{equation}
\subsubsection{Mellomakse-teoremet}
\label{subseq:tennis}
Mellomakse-teoremet, også kjent som tennisracket-teoremet (Tennis Racket Theorem) \cite{DAMME:1} sier at rotasjon om første og tredje prinsipielle akse er stabilt, mens rotasjon om andre akse er ustabil. Med prinsipiell akse mener vi de $3$ aksene legemet kan rotere om, og hver akse har sitt eget treghetsmoment $I$. Dette medfører at det stive legemet vil rotere stabilt om aksene med minst og størst treghetsmoment.\newline\newline
Vi sier at vi har en stabil rotasjon om en akse $j\ne i$ hvis
\begin{equation}
\label{eqn:stabil}
    \frac{d^2}{dt^2}\omega_i=-k\cdot\omega_i
\end{equation}
hvor $k$ er et positivt tall. Siden $\omega_i$ blir motvirket, så vil rotasjonen om akse $j$ holde seg stabil.\newline\newline
Vi sier at vi har en ustabil rotasjon om en akse $j\ne i$ hvis
\begin{equation}
\label{eqn:ustabil}
    \frac{d^2}{dt^2}\omega_i=k\cdot\omega_i
\end{equation}
Siden $\omega_i$ ikke blir motvirket, så vil den fortsette å øke, og rotasjonen om akse $j$ vil bli ustabil.\newline\newline
\paragraph{Bevis}
Her følger et uformelt bevis på hvorfor mellomakse-teoremet er sant. La oss si at vi har et stivt legeme med 3 rotasjonsakser med treghetsmoment $I_1, I_2$ og $I_3$ og $I_1>I_2>I_3.$ La vinkelhastighetene rundt de $3$ aksene være $\omega_1, \omega_2,$ og $\omega_3.$ Vi antar konstant dreieimpuls her, så dreiemomentet $\tau=0$. Da vil Eulers likninger \cite{EULER:1} gi oss
\begin{equation}
\label{eqn:euler1}
    I_1\frac{d}{dt}\omega_1=(I_2-I_3)\omega_2\omega_3
\end{equation}
\begin{equation}
\label{eqn:euler2}
    I_2\frac{d}{dt}\omega_2=(I_3-I_1)\omega_3\omega_1
\end{equation}
\begin{equation}
\label{eqn:euler3}
    I_3\frac{d}{dt}\omega_3=(I_1-I_2)\omega_1\omega_2
\end{equation}
Vi antar også at det er små vinkelhastigheter langs alle aksene, uavhengig av hvilken akse det stive legemet roterer rundt.\newline\newline
La oss først anta at det stive legemet roterer om akse $1.$
Hvis vi deriverer likning (\ref{eqn:euler2}) og bytter inn $\frac{d}{dt}\omega_3$ fra likning (\ref{eqn:euler3}), så får vi
\begin{equation}
    I_2I_3\frac{d^2}{dt^2}\omega_2=(I_3-I_1)(I_1-I_2)\omega_1^2\cdot\omega_2
    \implies \frac{d^2}{dt^2}\omega_2=-k\cdot\omega_2
\end{equation}
siden $I_1-I_2>0$ og $I_3-I_1<0.$ Vi kan vise tilsvarende for $\omega_3$ også. Siden vinkelhastighetene til de andre aksene blir motvirket, så oppnår vi en stabil rotasjon her. Dette kommer fra likning (\ref{eqn:stabil}). Vi kan også vise tilsvarende for rotasjon om akse $3.$\newline\newline
Lar det stive legemet rotere om akse $2.$ Hvis vi nå deriverer likning (\ref{eqn:euler1}), så får vi
\begin{equation}
    I_1I_3\frac{d^2}{dt^2}\omega_1=(I_2-I_3)(I_1-I_2)\omega_2^2\cdot\omega_1
    \implies \frac{d^2}{dt^2}\omega_1=k\cdot\omega_1
\end{equation}
siden $I_2-I_3>0$ og $I_1-I_2>0.$ Vi kan vise tilsvarende for $\omega_3$ også. Siden vinkelhastighetene til de andre aksene blir større, så oppnår vi en ustabil rotasjon her. Dette kommer fra likning (\ref{eqn:ustabil}).